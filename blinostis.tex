\documentclass{article}
\usepackage{graphicx} % Required for inserting images

\title{LaTeX}
\author{Daniil Belsky}
\date{September 2025}

\begin{document}

\maketitle
at the physical level, messages are forwarded be-
tween ostis-systems, generally physically located in
different nodes of the network, arbitrarily distant
from each other. This idea of separating the logical
and physical layers of communication is realized
within the concept of \textit{overlay networks} [17]. An overlay network is a virtual network whose structure
differs from the real communication network on
the basis of which this overlay network functions.
The idea of using \textit{corporate ostis-systems} as a basis for agents’ communication in such a network and
a repository of agents’ specifications and methods
provided by them can be considered as a new stage
of development of the idea of P2P agent platform
developed by the FIPA consortium [17]. The main
idea of such a platform is to provide all agents in the
network with the possibility of semantic search for
the services they need, as well as to search for agents
that possess the required services. Another function
of the platform is to support transparent communi-
cation between agents-customers and agents-owners
of services [17]. In general, an agent network may
have several such platforms, each responsible for
a different segment of the network, similar to the
way a \textit{corporate ostis-system} is responsible for a corresponding \textit{ostis-community}.
To implement the language of interaction between
ostis-systems, it is proposed to use the ideas of
wave programming [18], [19], as well as insertion
programming [20], [21] as a basis. Later variants of
the wave language theory development were called
Spatial Grasp Technology [18], [19], within which
Spatial Grasp Language is developed accordingly.
Implementing such interaction requires the develop-
ment of at least two levels of languages:
– transport layer defining the principles of record-
  ing SC-code constructions in some format conve-
  nient for network transmission. As a variant of
  such language it is proposed to use SCs-code [4];
– semantic level defining the content of messages
  transmitted over the network. The SCP language,
  which is the basic programming language for
  ostis-systems, is proposed to be used as a basis
  for such a language [4].
It is important to note that within the framework of
the proposed approach, the \textit{corporate ostis-system} acts as a common information resource and notifies the participants of the problem solving process about the events, but does not control the problem
solving process directly. Thus, it is not a question of
rigid \underline{imperative management}, but of more flexible
\underline{declarative}. This allows us to realize the advantages
of the sc-agent interaction mechanism in a shared
semantic memory [4], [11], such as modifiability
of the agent system, convenience of its design and
others.
\textbf{• Step 5: Develop the detailed architecture of the
multi-agent system}. At this stage, it is supposed to
clarify the principles of interaction between agents
and the environment, taking into account the previ-
ously clarified agent structure and the principles of
their interaction.
Implementation of the proposed approach assumes
that each ostis-system will include a communication
interface subsystem that will receive messages from
the external environment (from the \textit{corporate ostis-
system}), transform them into tasks for internal sc-
agents of the \textit{ostis-system}, and then transform the
result of their work into a response message and
send it to the corresponding recipient. An important
feature of such a subsystem is that it behaves as
a sc-agent when interacting with internal sc-agents
and communicates with other sc-agents according
to the same principles. This allows to ensure the
independence of the development and evolution of
such a subsystem from other components of the
ostis-system and to exclude the necessity to take
into account the fact of future interaction of the
ostis-system with other ostis-systems at the stage of
its design. In other words, an ostis-system solves
a subtask within a distributed collective of ostis-
systems just as if it were solving a problem explicitly
formulated, for example, by a user. This approach
greatly simplifies the design of ostis-systems col-
lectives, eliminating explicit dependencies between
them and the need to provide for the necessity of
collective problem solving in advance.
In turn, each \textit{corporate ostis-system}, when inter-
preting a \textit{particular method}, interacts with other
ostis-systems as if they were internal sc-agents of
this ostis-system. Thus, each \textit{corporate ostis-system}
includes an interface subsystem that converts events
in the memory of the \textit{corporate ostis-system} into
messages to other \textit{ostis-systems} and response mes-
sages from these \textit{ostis-systems} into corresponding
information constructs in the knowledge base of
the corporate ostis-system. This approach allows to
ensure the independence of \textit{corporate ostis-systems}
from other \textit{ostis-systems} participating in the prob-
lem solving process and to exclude the necessity to
provide in advance the necessity of collective prob-
lem solving not only when designing \textit{conventional
ostis-systems}, but also when designing corporate
ostis-systems. An illustration of this approach will
be given below.
From the point of view of the modern classification
of self-organization methods in multi-agent systems [17],
the proposed approach of agent interaction at the logical
level can be considered as a kind of mechanism based
on indirect interactions of organizational agents. Mecha-
23
nisms of this kind assume the absence of direct interac-
tion between autonomous entities composing the system,
but their interaction through a common environment,
which in the framework of the proposed approach is a
common semantic memory (both within the textit{individual
ostis-system} and within the \textit{collective ostis-system}).
V. Means of specification of next-generation intelligent
computer systems in the context of collective problem
solving
An important role in the \textit{Ecosystem} is played by a
rather detailed and unified specification of ostis-systems
included in the OSTIS \textit{Ecosystem}. Each ostis-system
included in the OSTIS \textit{Ecosystem} is subject to a number
of requirements [4], [8], the fulfillment of which is
necessary to ensure the principle possibility of collec-
tive problem solving, to increase the efficiency of the
evolution of OSTIS \textit{Ecosystem}and OSTIS \textit{Technology},
to reduce the requirements to the developers of ostis-
systems and the labor intensity of their development. The
most important of these requirements is the requirement
to ensure compatibility (both syntactic and semantic)
of each\textit{ostis-system} with others, and in particular with
the OSTIS \textit{Metasystem} containing the current version of
the OSTIS \textit{Standard}, and to continuously analyze and
maintain such compatibility.
At the same time, in order to organize problem solving
within OSTIS \textit{Ecosystem} it is additionally necessary to
have a detailed specification of functional capabilities of
each ostis-system and to ensure the relevance of such
specification in the process of evolution of this ostis-
system. This specification is part of the knowledge base
of \textit{corporate ostis-systems ostis-communities}, to which
the specified ostis-system belongs. If the \textit{ostis-system} is
not currently a part of any \textit{ostis-community}, the \textit{corporate
ostis-system} is the OSTIS \textit{Metasystem}.
The basis of the knowledge base of any ostis-system
is a hierarchical system of \textit{sc-models of subject domains}
and their corresponding formal \textit{ontologies} describing the
properties of entities studied within the specified subject
domains [4], [22]. Thus, the knowledge base of the \textit{cor-
porate ostis-system} contains sc-models of those subject
domains, on the automation of various activities in which
the corresponding \textit{ostis-community} is oriented. In order
to provide the possibility of automatic determination of
the collective of ostis-systems necessary for solving a
particular problem and clarifying the plan of solving
this problem, it is necessary to develop for each subject
domain the corresponding \textit{ontology of subject domain
problem classes and problem solving methods}. [4], [9].
The specified ontology includes a description:
• \textit{classes of problems} solved in the corresponding
  subject domain;
• \textit{methods} of problem solving corresponding to the
  specified \textit{classes of problems};
• \textit{skills} of problem solving corresponding to the speci-
  fied \textit{classes of problems}, i.e. \textit{methods}, supplemented
  by the description of \textit{sc-agents} implementing the
  specified \textit{methods} with the corresponding specifica-
  tion [4];
• \textit{method representation languages} specific to the sub-
  ject domain;
• \textit{strategies of problem solving} specific to the subject
  domain, i.e. meta-methods of forming other \textit{methods}
  of problem solving;
• and other entities, the description of which is        neces-
sary to organize problem solving processes within
the subject domain. For example, if there are several
methods of solving problems of the same class, it
is reasonable to describe their comparison in order
to be able to choose the method most suitable for
the current situation.
As mentioned earlier, the ontology presented in [15]
is proposed to be used as a basis for the content of the
general ontology of all possible problem classes solved
within the OSTIS \textit{Ecosystem}. Thus, the set of problem
classes described within a particular \textit{ontology of subject
domain problem classes and problem solving methods}
will specify some subset of problem classes from such
a top-level \textit{problem ontology}. Examples of describing
specific classes of problems and corresponding methods
of their solution using the example of neural network
methods of problem solving can be found in [23].
Thus, each ostis-system being a part of some ostis-
community should be specified using the concepts of
\textit{ontology of subject domain problem classes and prob-
lem solving methods} presented within the corresponding
\textit{corporate ostis-system}. In its turn, within each individual
ostis-system, this ontology can be further refined. Note
that the same methods (and, accordingly, skills) can
be duplicated between different ostis-systems, but the
information about it is explicitly recorded, which allows
us to take it into account when forming a problem solving
plan and determining the composition of participants of
the collective of ostis-systems taking part in the solution.
Accordingly, when adding ("registering") an ostis-
system to an ostis-community, the following steps must
be performed:
• Integrate the \textit{ontology of subject domain problem
classes and problem solving methods} into the cor-
responding ontology of the corporate ostis-system.
Thus, the \textit{corporate ostis-system} will receive infor-
mation about new problem classes and methods of
their solving, if there are any in the added ostis-
system;
• Using the obtained integrated ontology, generate
a specification of the added ostis-system in the
knowledge base of \textit{corporate ostis-system};
• When the functionality of a \textit{ostis-system} changes, it
must notify the \textit{corporate ostis-systems} of all \textit{ostis-
communities} of which this ostis-system is a part,
which in turn will lead to corresponding changes in
the knowledge bases of these corporate ostis-systems
and possibly to refinements of their corresponding
ontologies of subject domain problem classes and
problem solving methods. Note that this approach
has an advantage over many traditional approaches
to agent communication in multi-agent systems,
where for successful subsequent operation of the
system it is required to inform about the addition of
a new agent all agents already in the system, since
in the process of problem solving agents exchange
messages directly and must "know" each other.
The considered specification of ostis-systems within
the framework of OSTIS Ecosystem can be used not
only for organizing problem solving, but also for other
purposes, in particular, for implementing the idea of
component design of ostis-systems [24]. Besides, the
considered specification of ostis-systems is also necessary
for the developers of ostis-systems in order to understand
what capabilities are already presented within OSTIS
Ecosystem, within which ostis-communities, with the
developers of which ostis-systems it is necessary to
coordinate these or those components of the developed
system, and for solving a number of other design problem
solving.
VI. A general plan for solving a specific problem
within the next-generation intelligent computer systems
ecosystem
According to the proposed approach to problem solv-
ing within the OSTIS Ecosystem, solving a particular
problem generally involves the following steps:
• Problem formulation. In general, two options are
possible at this step:
– the initiator of problem solving is an ostis-system,
which is a part of OSTIS Ecosystem. In this
case, the problem formulation is placed in the
knowledge base of the corresponding corporate
ostis-system. To describe the problem formulation
at the first stage, both the top-level ontology
of subject domain problem classes and problem
solving methods (included in the OSTIS Standard
and, respectively, in the knowledge base of the
OSTIS Metasystem) and more particular ontology
of subject domain problem classes and prob-
lem solving methods corresponding to the ostis-
systems belonging to the given ostis-community
can be used.
– the initiator of problem solving is an external
cybernetic system, in particular a human user. In
this case, it is assumed that communication with
the OSTIS Ecosystem is carried out by a personal
ostis-assistant corresponding to this cybernetic
system. Thus, in this case, the task formulation
is placed in the knowledge base of the personal
ostis-assistant and then moved to the knowledge
base of the corporate ostis-system of the ostis-
community of which this personal ostis-assistant
is a member. If a user is a member of several
ostis-communities through his/her personal ostis-
assistant, then the problem of optimal selection
of the ostis-community within which it is most
expedient to start solving a problem becomes
relevant. At the same time, the proposed approach
to decentralized problem solving in general does
not depend on which corporate ostis-system the
problem formulation initially enters, it affects
only the total time of problem solving.
Thus, as a result of this step, in any case, the
problem formulation enters the knowledge base of
some corporate ostis-system (in general, not nec-
essarily that corporate ostis-system, which will act
as a communication environment in the process of
solving this problem).
• Determining the set of ostis-systems to be in-
volved in problem solving. In general, it may be
sufficient to involve only ostis-systems representing
one ostis-community, or a set of ostis-systems be-
longing to different ostis-communities. The specific
mechanism of this stage requires clarification, but
the following principles are suggested as its basis:
– the initiator of this stage is the corporate ostis-
system whose knowledge base contains the corre-
sponding problem formulation. For this purpose,
the specified corporate ostis-system interacts with
other corporate ostis-systems, if necessary in-
volving corporate ostis-systems of a higher level.
Development of a protocol for such interaction is
an actual task;
– the key role at this stage is played by the pre-
viously discussed ostis-systems specifications de-
scribing classes of problems, methods of their
solving, etc;
– in the process of performing this stage, the initial
problem formulation may be refined taking into
account particular ontologies of subject domain
problem classes and problem solving methods.
• Definition (selection) of corporate ostis-system,
which will be the communication environment for
solving the currently formulated problem solving
task. The principles of such a selection have been
discussed above.
• Formation of a problem solving plan. At this
stage of development of the theory of decentralized
problem solving within the OSTIS Ecosystem, we
will assume that the solution plan of a particular
problem is formed, stored and refined entirely within
the corresponding corporate ostis-system. In general,
we can talk about the possibility of distributed

\section{Introduction}

\end{document}
